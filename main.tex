\documentclass{beamer}  
\usepackage[UTF8,noindent]{ctexcap}  
\usetheme{CambridgeUS} 
\usepackage{textpos}
\usepackage{shanghaitech}

\setAuthorName{Zhixin Piao}
\setAuthorEmail{piaozhx6@gmail.com}
\setInstitudeName{ShanghaiTech University}

\begin{document}
\makeTitlePage


\section{Android系统架构}  
\subsection{我的架构}

\begin{frame}{Android简介}{Android的发展和历史}  
Android并不是Google创造的,而是由Android公司创造的,该公司的创始人是Andy Rubin。该公司后来被Google收购。  
在讲Android之前我们先看看其他的操作系统:  
\begin{itemize}  
\item Symbian  
\begin{itemize}  
\item Symbian  
\item BlackBerry  
\item iPhone
\end{itemize}  
\item BlackBerry  
\item iPhone
\end{itemize}  
\end{frame}  

\section{Android安全} 
\subsection{我的架构2}
\begin{frame}  
\begin{block}{勾股定理}  
直角三角形的斜边的平方等于两直角边的平方和。  
可以用符号语言表述为:设直角三角形ABC,其中$\angle C=90^\circ$则有  
\begin{equation}  
AB^2=BC^2+AC^2  
\end{equation}  
\end{block}  
\end{frame}

  
\section{总结}
\begin{frame}
  在讲Android之前我们先看看其他的操作系统:  
\end{frame}  


\end{document}  
